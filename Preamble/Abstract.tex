\unnumberedsection{Abstract}
\section*{Abstract}
\subsection*{\thesistitle}
    
    The popularity of cloud services is growing exponentially, on the one hand large technology multinationals are investing millions in improving their infrastructure and services. On the other hand, users are increasingly dependent on these services. However, we must not forget that today we have at our disposal, devices for everyday use that exceed the performance of computers that brought humans into space during the last century. These devices are part of the periphery, the edge layer, and are often diverse in nature. We want to make use of this computing power that would otherwise be wasted.The novelty results are twofold: an open-source orchestrator that handles heterogeneous computing agents; an open-source front-end application to interact with these.
    
    \vskip\baselineskip
    \begin{keyword}
        Cloud computing, Edge computing, Heterogeneous computing, Distributed systems, Orchestration
    \end{keyword}
    
    \vskip\baselineskip
    La popularitat dels serveis al núvol està incrementant de forma exponencial, per un costat les grans multinacionals tecnològiques inverteixen milions en millorar les seves infraestructures i serveis. Per l'altre costat, els usuaris cada cop depenem més d'aquests serveis. No obstant, no ens hem d'oblidar que avui en dia tenim al nostre abast, dispositius d'ús quotidià que superen en rendiment als ordinadors que van portar l'ésser humà a l'espai durant el segle passat. Aquests dispositius formen part de la perifèria, la capa de l'\textit{edge} i sovint tenen diversa naturalesa. Volem donar-li un ús a aquesta capacitat computacional que altrament estaria desaprofitada. Els resultats de la novetat són dobles: un orquestrador de codi obert que maneja agents informàtics heterogenis; una aplicació frontal de codi obert per interactuar amb aquests.
    
    \vskip\baselineskip
    \begin{keyword}
        Informàtica núvol, Computació frontera, Computació heterogènia, Sistemes distribuïts, Orquestració
    \end{keyword}
    
    
    
    \vskip\baselineskip
    La popularidad de los servicios en la nube está incrementando de forma exponencial, por un lado las grandes multinacionales tecnológicas invierten millones en mejorar sus infraestructuras y servicios. Por otro lado, los usuarios dependen cada vez más de estos servicios. Sin embargo, no debemos olvidar que hoy en día tenemos a nuestro alcance, dispositivos de uso cotidiano que superan en rendimiento a los ordenadores que llevaron al ser humano al espacio durante el siglo pasado. Estos dispositivos forman parte de la periferia, la capa del \textit{edge} y con frecuencia tienen diversa naturaleza. Queremos darle un uso a esta capacidad computacional que de otro modo estaría desperdiciada. Los resultados novedosos son dos: un orquestador de código abierto que maneja agentes informáticos heterogéneos; una aplicación front-end de código abierto para interactuar con estos.
    
    \vskip\baselineskip
    \begin{keyword}
        Computación nube, Computación frontera, Computación heterogénea, Sistemas distribuidos, Orquestración
    \end{keyword}