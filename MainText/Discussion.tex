\section{Discussion}
    In light of the above, the Yako orchestrator works sublimely with the existing simple heuristic and deploys applications across the nodes of the cluster. Before examining the final set of nodes, the orchestrator ignores those that do not match the requirements. As shown in the previous section examples, a deployment of an application to IoT devices will discard the other non-IoT.
    The current algorithm examines the viability to distribute an application to a node at deploy time, depending on the current state of its resources. However, more advanced and complete heuristics, as described in the previous paragraphs, or AI based algorithms could be taken into consideration to further boost its smartness. For instance, using an AI-based technique could use stored information of the previous runs and predict the feasibility of the new deployment on that node.
    
    Regarding the UI, currently the front-end application pages, (i) Dashboard (ii) Cluster Graph, only display basic telemetry information in a text-based manner. Adding the already designed expanded right panel view (see figure \ref{fig:expanded_info_panel_ui}), could further improve the insight of the cluster state, for instance, its working agents and the resources being utilized. As stated in \cite{stark_why_2020}, data visualization can provide users, in this case the system administrator, a clearer idea of what the information means by using maps or graphs.
    New components are under development, such as uploaded applications management and front-end application settings. Such changes are publicly available on GitHub repository \cite{chen_yakoui_2022}.
    
    During the internal testings, the Yako platform performed correctly, all its nodes, (i) YakoMaster, (ii) YakoAgent, (iii) YakoAgent (IoT), worked with full availability with no disconnections.
    Nevertheless, for the application to be production-ready more testing coverage should be undergone. This reinforces the need of considering using a bigger scale heterogeneous distributed system and more advanced testing techniques \cite{neeru360_software_2021}, including QA, performance monitoring, stress loading and use case tests.